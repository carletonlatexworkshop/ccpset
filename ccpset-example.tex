% Example for Carleton problem sets
% Author: Andrew Gainer-Dewar, 2013
% This work is licensed under the Creative Commons Attribution 4.0 International License.
% To view a copy of this license, visit http://creativecommons.org/licenses/by/4.0/ or send a letter to Creative Commons, 444 Castro Street, Suite 900, Mountain View, California, 94041, USA.
\documentclass[twoside]{article}
\usepackage{ccpset}

% The Latin Modern font is a modernized replacement for the classic
% Computer Modern. Feel free to replace this with a different font package.
\usepackage{lmodern}

% These commands tell the ccpset package about your document.
% They are used to set up the header.
\title{Example problem set}
\author{Lori T.~Jeremiah \and Ray B.~Langstrom} % The \and command handles putting the names on separate lines.
\date{February 31, 2014}
\prof{Professor M.~Holmes}
\course{HIST-314: History of Magic}

\begin{document}
% Calling \maketitle tells the package to print out the title using
% the information you configured above.
\maketitle{}

% Use of sections is optional.
% If you don't want to have them, just omit lines like these.
\section*{Chapter 4}

% The pset environment creates a list and configures the
% \problem{} and \exercise{} commands.
% All your problems should be inside a pset environment, but
% you can have more than one if you need to.
\begin{pset}
  % And here's where the fun begins.
  % For each problem, use a \problem{} command, with the problem number
  % as an argument. (You can also use the \exercise{} command, which just
  % uses the word "Exercise" instead of "Problem".)
  \problem{2(a)}
  % If you want to include the problem statement, you can do so using the
  % statement environment.
  \begin{statement}
    Prove that there are infinitely many primes numbers.
  \end{statement}
  % The proof environment can be used to format solutions that are proofs.
  % This example also demonstrates the use of inline math.
  \begin{proof}
    Suppose that $\mathbf{p} = \cbrac{p_{1}, p_{2}, \dots, p_{n}}$ is a finite set of primes.
    Then $1 + \prod_{i = 1}^{n} p_{i} = 1 + p_{1} p_{2} \cdots p_{n}$ is not divisible by any $p_{i}$, so there is a prime not in $\mathbf{p}$.
    Thus, no finite set contains all primes.
  \end{proof}

  % If you don't want to type up the problem statement, you can just dive right in!
  \problem{17.2}
  It follows from Theorem 2.7.14-a that
  % Here, we'll use an align* environment to set several equations without numbers.
  \begin{align*}
  \sin A \cos B &= \frac{1}{2} \pbrac*{\sin \pbrac*{A-B} + \sin \pbrac*{A+B}}, \\
  \sin A \sin B &= \frac{1}{2} \pbrac*{\sin \pbrac*{A-B} + \cos \pbrac*{A+B}}, \\
  \intertext{and finally that} % The \intertext{} command inserts paragraph text into multiline display math
  \cos A \cos B &= \frac{1}{2} \pbrac*{\cos \pbrac*{A-B} + \cos \pbrac*{A+B}}.
\end{align*}
\end{pset}

\section*{Worksheet 4}
% If you start a new section, it needs to be *outside* the pset environment,
% due to complications with the way LaTeX handles lists.
\begin{pset}
  \exercise{19}
  We can conclude the following:
  % The theorem-like environments all work in problem sets, in case you need them.
  \begin{theorem*}
    There are $2^{\binom{n}{2}}$ labeled graphs on $n$ vertices.
  \end{theorem*}

  \exercise{2-i}
  The results of this computation appear in \cref{tab:example}.

  % Here, we put LaTeX's table features to use.
  \begin{table}[htb]
    \centering
    \begin{tabular}{c c}
      \toprule
      $x$ & $f(x)$ \\ \midrule
      $1$ & $1$ \\
      $2$ & $\pi$ \\
      $3$ & $\ln 4$ \\
      \bottomrule
    \end{tabular}
    \caption{Results of a calculation}
    \label{tab:example}
  \end{table}
\end{pset}
\end{document}